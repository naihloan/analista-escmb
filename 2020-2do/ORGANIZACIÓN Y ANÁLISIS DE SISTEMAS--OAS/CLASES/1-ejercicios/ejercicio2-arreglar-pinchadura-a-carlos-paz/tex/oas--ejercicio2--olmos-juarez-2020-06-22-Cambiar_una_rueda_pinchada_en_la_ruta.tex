\documentclass[]{article}
\usepackage{lmodern}
\usepackage{amssymb,amsmath}
\usepackage{ifxetex,ifluatex}
\usepackage{fixltx2e} % provides \textsubscript
\ifnum 0\ifxetex 1\fi\ifluatex 1\fi=0 % if pdftex
  \usepackage[T1]{fontenc}
  \usepackage[utf8]{inputenc}
\else % if luatex or xelatex
  \ifxetex
    \usepackage{mathspec}
  \else
    \usepackage{fontspec}
  \fi
  \defaultfontfeatures{Ligatures=TeX,Scale=MatchLowercase}
\fi
% use upquote if available, for straight quotes in verbatim environments
\IfFileExists{upquote.sty}{\usepackage{upquote}}{}
% use microtype if available
\IfFileExists{microtype.sty}{%
\usepackage{microtype}
\UseMicrotypeSet[protrusion]{basicmath} % disable protrusion for tt fonts
}{}
\usepackage[unicode=true]{hyperref}
\hypersetup{
            pdfborder={0 0 0},
            breaklinks=true}
\urlstyle{same}  % don't use monospace font for urls
\IfFileExists{parskip.sty}{%
\usepackage{parskip}
}{% else
\setlength{\parindent}{0pt}
\setlength{\parskip}{6pt plus 2pt minus 1pt}
}
\setlength{\emergencystretch}{3em}  % prevent overfull lines
\providecommand{\tightlist}{%
  \setlength{\itemsep}{0pt}\setlength{\parskip}{0pt}}
\setcounter{secnumdepth}{0}
% Redefines (sub)paragraphs to behave more like sections
\ifx\paragraph\undefined\else
\let\oldparagraph\paragraph
\renewcommand{\paragraph}[1]{\oldparagraph{#1}\mbox{}}
\fi
\ifx\subparagraph\undefined\else
\let\oldsubparagraph\subparagraph
\renewcommand{\subparagraph}[1]{\oldsubparagraph{#1}\mbox{}}
\fi


% \usepackage{enumerate}
% \usepackage[shortlabels]{enumitem}
\renewcommand{\labelenumii}{\theenumii}
\renewcommand{\theenumii}{\theenumi.\arabic{enumii}.}

% \newlist{legal}{enumerate}{10}
% \setlist[legal]{label*=\arabic*.}

\date{}
\begin{document}

\emph{\emph{Cambiar una rueda pinchada en la ruta}}

Este ejercicio sirve para señalar los pasos sucesivos necesarios para reaccionar a cambiar una rueda pinchada en camino de ruta.
Se atiende a la secuencia y causalidad de un paso a otro. En el caso de la pinchadura en este ejemplo se da en un paso en el medio de la ruta: el punto número 10, a la altura de Malagueño. 
Ese mismo paso puede darse en cualquier punto del camino. 
El conjunto de procedimientos es equivalente a si se pincha en un lugar u otro
% la rueda en cualquier otro punto del camino a si se pincha en el medio de la ruta 
como indica el ejemplo.

\begin{enumerate}
\def\labelenumi{\arabic{enumi}.}
\item Salir de mi casa
\item Entrar al auto
\item Arrancar el auto
\item Acelerar
\item Tomar la ruta de circunvalación saliendo del arco de Córdoba % \end{enumerate}
%     \begin{itemize}
    \begin{enumerate}
    \def\labelenumi{\arabic{enumi}.}
    \item O bien tomar la ruta de Av Sabatini por el centro
    \end{enumerate}
%     \end{itemize}

\item Llego al peaje
\item Abono la tasa del peaje
\item Salgo del peaje
\item Paso por el control de caminera
\item Sigo recto por la carretera
%
  %%%%% INICIO DE PROBLEMA
      \begin{enumerate}
      \item A la altura de Malagueño, noto un ruido extraño en la rueda % \end{enumerate}
      \item Procedo a frenar el auto
%     \begin{itemize}
    \begin{enumerate}
    \def\labelenumi{\arabic{enumi}.}
    \item Prendo las balizas
    \item Verifico por el espejo retrovisor si no viene algún auto
    \item Disminuyo la velocidad
    \item Orillo el auto al costado de la ruta
    \item Detengo el auto
    \end{enumerate}
%     \end{itemize}

\item Me bajo a revisar cual es el problema
\item Encuentro la rueda delantera izquierda pinchada
\item Abrir el baúl del auto
\item Sacar las herramientas y llanta de repuesto
\item Colocar triángulo de balizas fuera del auto a 3 metros hacia el tráfico
\item Dejar los materiales junto a la rueda
\item Desajustar tuercas parcialmente
\item Colocar gato
\item Elevar auto
\item Desajustar tuercas totalmente
\item Quitar tuercas
\item Quitar rueda
\item Colocar rueda de auxilio
\item Colocar tuercas
\item Ajustar tuercas levemente
\item Bajar auto con el gato
\item Quitar gato
\item Ajustar tuercas
\item Juntar herramientas, rueda, gato
\item Guardar herramientas, rueda, gato en el baúl
\item Procedo a subir al auto
% \end{enumerate}

    \begin{enumerate}
    \def\labelenumi{\arabic{enumi}.}
    \item Subir al auto
    \item Sacar las balizas
    \item Encender el auto
    \item Mirar en el retrovisor izquierdo si vienen autos
    \end{enumerate}
\item Salir andando
% \item Acelerar
% \begin{enumerate}
% \def\labelenumi{\arabic{enumi}.}
\item
  Estar atento a que no haya ningún otro ruido extraño en la rueda
      \begin{enumerate}
    \def\labelenumi{\arabic{enumi}.}
    \item Si hay ruido volver a proceder a frenar el auto a revisar
    \end{enumerate}
      \end{enumerate}
  %%%%% CIERRE DE PROBLEMA
%
\item
  Llegar a Carlos Paz
\item
  Disfrutar de la tarde en un bar del centro de la ciudad
\end{enumerate}

\end{document}
