% \hypertarget{unidad-x}{%
% \section{UNIDAD X:}\label{unidad-x}}

\twocolumn
\hypertarget{calidad-q}{%
\section{Calidad [Q]}\label{calidad-q}}

% d5sSX9BB3A

\begin{quote}
AFEDUC PORMANFIASE
\end{quote}

Qué es calidad? Como gente de sistemas queremos dar soluciones o
productos a un cliente, la confianza en la garantía del producto que le
vamos a dar, que sea una solución confiable, y es un producto que se va
a adecuar a las necesidades del cliente, mientras se cumpla la
adecuación a un cierto estándard. Para eso existen las normas de
calidad. Por ejemplo, si hablamos de software, podemos hablar de las
normas de calidad de desarrollo de software, que es la ISO 25010, que
contempla cuáles son esos estándares a la hora de dar un desarrollo de
software de calidad.

Ahora vamos a nombrar cuáles son esas facetas, para después entrar en
más detalle de cada una.

\begin{itemize}
  \setlength\itemsep{-1em}
\item
  Adecuación Funcional {[}AF{]}
\item
  Eficiencia de Desempeño {[}ED{]}
\item
  Usabilidad {[}U{]}
\item
  Compatibilidad {[}C{]}
\item
  Portabilidad {[}POR{]}
\item
  Mantenibilidad {[}MAN{]}
\item
  Fiabilidad {[}FIA{]}
\item
  Seguridad {[}SE{]}
\end{itemize}

\begin{center}\rule{0.5\linewidth}{0.5pt}\end{center}

\hypertarget{adecuaciuxf3n-funcional-af}{%
\subsection{Adecuación Funcional
{[}AF{]}}\label{adecuaciuxf3n-funcional-af}}

La adecuación funcional hace a aspectos comprobables de software, y
siempre con cotejar lo que se negoció y se estableció previamente. No se
compara con algo estándard, sino que se compara con algo definido en la
etapa de negociación. Con la adecuación funcional se va a comprobar si
el sistema de software cumple con todos los requerimientos que se habían
definido, y si lleva adelante todas las funciones que se había
establecido que debían llevar.

En este paso hay que analizar varios aspectos:

\hypertarget{completitud}{%
\paragraph{Completitud}\label{completitud}}
  Que se cumplan todos y cada uno de los requerimientos definidos previamente.
% % % % % % % % % % % % % % % % % % % % % % % % % % % % % % % % % % % % % % %
\hypertarget{correccion}{%
\paragraph{Corrección}\label{correccion}}
 Que el sistema funcione en términos y niveles de
  precisión definidos previamente. Se puede pensar por ejemplo con la
  precisión de los números, si se define que los resultados deben dar un
  detalle de hasta 3 decimales.
% % % % % % % % % % % % % % % % % % % % % % % % % % % % % % % % % % % % % % %
\hypertarget{pertinencia}{%
\paragraph{Pertinencia}\label{pertinencia}}
%   \paragraph{Pertinencia.}
  Es la adecuación de estas funcionalidades según
  lo definido previamente. Que el sistema sea apropiado para lo que se
  estableció.
% % % % % % % % % % % % % % % % % % % % % % % % % % % % % % % % % % % % % % %
\hypertarget{eficiencia-de-desempeuxf1o-ed}{%
\subsection{Eficiencia de Desempeño
{[}ED{]}}\label{eficiencia-de-desempeuxf1o-ed}}

Siempre que se habla de eficiencia o desempeño, se habla de recursos.
Cuando se habla de un producto de software, el desarrollador tiene que
saber qué recursos va a necesitar el software para funcionar en las
condiciones previstas con el cliente. También vamos a hablar del
desempeño, por ejemplo la respuesta temporal, qué tiempo de respuesta va
a tener el software, y si ese tiempo de respuesta cumple o no con los
requisitos, ya que tiene que estar determinado y conciente el
desarrollador de software.

\hypertarget{capacidad}{%
\paragraph{Capacidad}\label{capacidad}}

Como parte del desempeño está la capacidad, de cuál es el límite máximo
del que es capaz el software. Sabemos en qué condiciones va a operar, y
en qué funciones. Dicho en criollo: el software no se tiene que quedar
corto. Si mucha gente entra al mismo tiempo a una plataforma, el sistema
tiene que tener capacidad de respuesta ante los usuarios. Y acá no hay
que guiarse solamente por un estándard, sino pautarlo todo de antemano.
Se puede decir que la reacción ante una solicitud sea entre 1 segundo y
1,3 segundos. Entonces el sistema se tiene que adecuar a ese tiempo.

\begin{quote}
{\slshape
(Paréntesis sobre Calidad) Cuando hablamos de calidad, nos sirve en el
mundo laboral para saber qué cosas preguntarle al cliente. Saber qué
tipos de respuesta esperan del software, qué funciones esperan que se
cumplan para que opere de manera correcta, qué capacidad necesitan. Y de
acuerdo con las expectativas, darle las funcionalidades según lo
esperado, con los límites adecuados a las necesidades del cliente. Cada
uno de los elementos que surgen dentro de calidad, tienen que surgir
entonces las preguntas para el cliente. Y todo lo que se negocie tiene
que quedar plasmado en documentos, revisados y firmados por el cliente.
Lo que hay que evitar es que te corran el alambrado, y si no se definió
bien que el sistema tenía que hacer cierta cosa, entonces si el cliente
pide agregados, tenemos que tener un fundamento legal de qué es lo que
estaba incluido en los planes, y qué es lo que quedaba afuera, acorde a
lo pautado.
}
\end{quote}

\hypertarget{usabilidad-u}{%
\subsection{Usabilidad {[}U{]}}\label{usabilidad-u}}

En este apartado, al referirnos a usabilidad se está teniendo en cuenta
la experiencia del usuario {[}UX{]}. Hay varios componentes de la
experiencia de usuario {[}UX{]}:

\hypertarget{accesibilidad}{%
\paragraph{Accesibilidad}\label{accesibilidad}}
Si un usuario quire usar el software, si tiene
  algún tipo de discapacidad, lo puede usar sin tener ningún
  inconveniente. Lo mismo que significa accesibilidad para un edificio,
  para la arquitectura, para la ingeniería: también acá la accesibilidad
  se refiere a dar posibilidad de uso a todas las personas,
  independiente a si tienen algún tipo de incapacidad, o discapacidad.
\hypertarget{aprendizabilidad}{%
\paragraph{Aprendizabilidad}\label{aprendizabilidad}}
 El software tiene que poder ser aprendido.
  Cuando el usuario entra al software, el entorno tiene que dar
  herramientas para que el usuario pueda resolver lo que necesita hacer
  con el software.
  También el sistema tiene que ayudar a prevenir los
  errores del usuario, aportar a que los errores no se cometan.
\hypertarget{estetica}{%
\paragraph{Estética}\label{estetica}}
  Y también entran temas de estética. Se busca que el sistema sea
  agradable a la vista, tiene que ser algo que no genere un rechazo, un
  problema a la hora de que el usuario ejecute su tarea. Cuando un
  software llega a un nivel de calidad, en lo que hace a usabilidad, nos
  referimos a que la Interfaz de Usuario {[}UI{]} invita al usuario a
  que use ese programa, o sea que se vuelve fácil de aprender, fácil de
  aprender y fácil de usar una vez aprendido, sin complejidades
  innecesarias del lado del usuario. El usuario también tiene que
  reconocer que el software es apropiado. Acá se vuelve al tema de la
  adecuación, pero no de cara al cliente, que puede ser el dueño de la
  empresa, sino para el usuario final. Ante un home banking que se
  entienda por ejemplo que sirve para hacer transacciones, y no que la
  impresión sea que no sirve para nada, porque entra el usuario y puede
  ver en la pantalla que hay un título que indica: ``Transacciones''. Se
  puede decir que el software es fácil de usar, y es técnicamente
  correcto decirlo así porque son las facilidades necesarias para el
  usuario.

\hypertarget{compatibilidad-c}{%
\subsection{Compatibilidad {[}C{]}}\label{compatibilidad-c}}


\hypertarget{coexistencia}{%
\paragraph{Coexistencia}\label{coexistencia}}
La palabra clave que va con la compatibilidad
  es la coexistencia. Para afrontar hacer un software, no podemos pensar
  que el software va a estar solo en una esquina y va a funcionar solo
  por su cuenta, sino que existen miles de software que interactúan, y
  que actúan a la par con ese software que usa la empresa. Todas esas
  herramientas tienen que poder coexistir pacíficamente, sin que esta
  coexistencia implique un problema, sea tanto para el software del que
  nos ocupemos, como de los otros que lo colindan.
\hypertarget{intercambios}{%
\paragraph{Intercambios de Información}\label{intercambios}}
El software tiene que poder ser
  capaz de interactuar con los sistemas que lo rodean, hacer
  intercambios de información con los otros sistemas. Si un sistema
  necesita enviar o extraer información, el software tiene que ser capaz
  de hacerlo sin problemas.


\begin{quote}
\textbf{Cuándo salta la incompatibilidad de un software?}
\slshape{
Puede pasar
por ejemplo que la información que se busca y se da dentro de dos
sistemas, tengan formatos diferentes, lo que los hace incompatibles. Por
ejemplo, tengo algo en Corel, y lo exporto a Adobe, y no lo reconoce.
Esto pasa así, y son competencia. Son incompatibles intencionalmente. Lo
que buscamos en calidad de software, es que esto no pase. Cada vez los
software deben ser más compatibles, y más todavía si son de una misma
empresa. Todos los software de una misma empresa deben ser compatibles
entre sí. Por eso es importante a la hora de desarrollar un nuevo
software, entender cuáles son los otros software que lo van a rodear, y
ser capaces de dar los input/output necesarios para que funcionen en
conjunto. No es algo que se pueda evitar, o ignorar.}
\end{quote}

\hypertarget{portabilidad-por}{%
\subsection{Portabilidad {[}POR{]}}\label{portabilidad-por}}

El software debe tener una capacidad de ser instalado o desinstalado en
un entorno. Debe ser posible, así como se instala el software en un
entorno de desarrollo, bajo ciertas condiciones de productividad; tengo
que ser capaz de desinstalarlo y llevarlo a otro entorno, y que funcione
de la misma manera, sin perder información, que queden como existen los
estados de las entidades, que siga funcionando a pesar de haber sido
trasladado. Por eso también el software debe tener la capacidad de ser
re-emplazado.

\hypertarget{adaptabilidad}{%
\paragraph{Adaptabilidad}\label{adaptabilidad}}
El producto debe ser posible de funcionar en
diferentes condiciones de hardware, software, servidores, y operaciones.

\hypertarget{mantenibilidad-man}{%
\subsection{Mantenibilidad {[}MAN{]}}\label{mantenibilidad-man}}

El sistema no puede ser considerado como un sistema que va a ser único y
que se va a mantener en el estado en el que lo entregamos para toda la
vida. Es un sistema que tiene que estar preparado para evolucionar, para
cambiar, para solucionar problemas.

\hypertarget{modularidad}{%
\paragraph{Modularidad}\label{modularidad}}
El sistema tiene que estar separado en módulos,
  donde cada módulo remite a una funcionalidad específica. El principal
  objetivo al hacer esto es lograr que un cambio en un módulo, tenga el
  menor impacto posible en el resto. Esto también nos permite, a la hora
  de querer construir otro software, se van a poder usar algunos módulos
  en esa construcción, es decir que pueden ser reutilizados. // Todos
  los ítems de mantenibilidad se desprenden de la modularidad. Esta idea
  viene de la programación orientada a objetos {[}POO{]}: diferente a la
  programación orientada a procesos, donde se arma todo en un solo
  paquete. En la POO cada clase tiene responsabilidades únicas.

\hypertarget{analizabilidad}{%
\paragraph{Analizabilidad}\label{analizabilidad}}
Si quiero ver una sola función del software,
  debe ser posible aislarla para observarla y entenderla.

\hypertarget{modificable}{%
\paragraph{Capacidad de Ser Modificado}\label{modificable}}
Se puede modificar un sector,
  sin necesidad de hacer cambios en todo el sistema.

\hypertarget{testeable}{%
\paragraph{Capacidad de Ser Probado}\label{testeable}}
La mantenibilidad tiene que
  garantizar la posibilidad de probar partes del software sin necesidad
  de hacer andar el software entero cada vez que se quiera probar algo.

\hypertarget{fiabilidad-fia}{%
\subsection{Fiabilidad {[}FIA{]}}\label{fiabilidad-fia}}

Esto se refiere a la confiabilidad del sistema. Al igual que la
eficiencia de desempeño y que con la adecuación funcional, todas las
cotejaciones van a ser con respecto a cuestiones definidas previamente.
Es decir, que se van a definir previamente en qué condiciones
específicas el software va a funcionar de manera correcta.

En la fiabilidad se va a definir, por ejemplo, que en ciertos horarios
el software tiene que funcionar, tiene que ser confiable, tiene que
poder llevar adelante todas sus funciones, y en determinadas condiciones
específicas de ejecución. Acá también se habla de la tolerancia a
fallos. El sistema tienen que tener una protección de la información, y
de los estados que maneja que ante un fallo de hardware, por ejemplo, si
se reinicia el sistema, puedan estar las entidades en los estados en que
estaban, y la información guardada. Que no se quede a pata.

\hypertarget{seguridad-se}{%
\subsection{Seguridad {[}SE{]}}\label{seguridad-se}}

Último, pero no menos importante.


\hypertarget{confidencialidad}{%
\paragraph{Confidencialidad}\label{confidencialidad}}
En un primer momento, hablamos más de la
  confidencialidad de la información: si una persona ingresa al software
  y da datos sensibles, de la intimidad de una persona, que no está
  acordado entre las partes que esos datos sean compartidos a terceros,
  entonces el sistema debe garantizar esa confidencialidad, que no se
  vaya a compartir con otros usuarios. Se tiene que proteger la
  información del usuario y sobretodo su identidad, cuidando ante los
  robos de identidad.

\hypertarget{noaccess}{%
\paragraph{Prevenir accesos sin autorización}\label{noaccess}}
  El sistema debe prevenir
  accesos, o modificaciones no autorizadas a datos o programas del
  ordenador. Puede haber agentes que busquen extraer información para
  venderla, o hacer algún uso malicioso de la información.

\hypertarget{huellas}{%
\paragraph{Registros: Pensar en la huella}\label{huellas}}
 Todo hecho que suceda adentro
  del software tiene que quedar registrado, para determinar quién fue el
  responsable de los hechos: esta huela sirve para rastrear problemas, y
  dar soluciones.

\hypertarget{norepudio}{%
\paragraph{Capacidad de no repudio}\label{norepudio}}
  Si un usuario hace algo que afecta a
  otra persona, sea intencionalmente o no, no se puede desconocer el
  hecho, queda atado a quien lo haya ejecutado. El software debe
  registrar quién hizo qué acción.


\hypertarget{cierre-de-calidad-q}{%
\subsection{Cierre de Calidad {[}Q{]}}\label{cierre-de-calidad-q}}

La calidad no debe verse como algo pesado a cumplir. No es algo que se
hace al final. Es algo que se va dando cotidianamente. No es una tarea
extra para cumplir, sino que es una herramienta que asiste en
desarrollar un software funcional.

La calidad puede orientar en qué tipos de preguntas hacerle a un
cliente: qué tipo de seguridad necesitás, cuáles son los usuarios, cómo
es su identidad, cómo es el tipo de dato que manejan.

La calidad también indica los parámetros bajo los cuales se puede
acordar que hemos terminado de desarrollar un software. La calidad
define qué tan lejos, o qué tan cerca, estamos de cumplir con los
objetivos. Es una manera de anticiparse a las entregas, sabiendo qué es
lo que se acordó, sin que aparezcan elementos inesperados ni nuevos
pedidos. Se deja de antemano todo especificado, cuándo es que se termina
de armar ese software.

\onecolumn
